% simple.tex - A simple article to illustrate document structure.

% preamble

\documentclass{article}
%% \usepackage{times}
\usepackage{latexsym}
\usepackage{url}
\usepackage{hyperref}
\hypersetup{colorlinks=true}
\usepackage{graphicx}

\begin{document}

% top matter

\title{Artificial Intelligence Homework, Spring 2017}
\author{Turcu Gabriel-Virgil }
\date{Computer Science English Section, Second Year, Group 10206B}
\maketitle
\pagebreak



% sections
\section{Problem statement}

Your task is to develop an application to evaluate the performance of an heuristic
search algorithm.
We consider Recursive best-first search – RBFS. In what follows we present
the description:
“This algorithm uses the f limit variable to keep track of the f-value of the
best alternative path available from any ancestor of the current node. If the
current node exceeds this limit, the recursion unwinds back to the alternative
path. As the recursion unwinds, RBFS replaces the f-value of each node along
the path with a backed-up value – the best f-value of its children. In this way,
RBFS remembers the f-value of the best leaf in the forgotten subtree and can
therefore decide whether it’s worth reexpanding the \mbox{subtree at some later time.}”
\linebreak
\linebreak
P2) Top-down proof with SLD resolution. Here you can vary the number n of
propositional symbols, as well as the number k of rules.

%%This document is about \LaTeX{}~\cite{lamport94}
\pagebreak

\linebreak


\section{Application Design}

\begin{itemize}
\item
The pseudocode used was the pseudocode given in the problem statement for the RBFS function.

\item
The high level architectural overview of the application.
\begin{itemize}
  \item The application uses one recursive best first search function to solve the top-down proof with SLD resolution.
  
  \item The main containers used in the function and outside the function are represented by :
  \begin{itemize}
  \item frontConnections : this STL vector of vectors of int is used to hold connections leading downwards in our graph.
  \item toProve : this vector is used to hold the nodes that will need to be proven as connected to the "yes" node.
   
  \item backConnections : this vector is used to hold the parent for each node.
  
  \item completed : this vector is 1 if the node can be marked as completed and 0 if it is not completed yet.
  
  \item initialState : this deque that holds vectors is the initial state the RBFS will get at its first call.
	\end{itemize}
  
\end{itemize}

\item
\mbox{The Specification of the input.}\linebreak
\linebreak
\item This is how the input will be given in the test file:
\begin{enumerate}	\item First number represents the number of definitions. For each of the definitions we will have:
	\begin{enumerate} \item[1.1] The number of parameters.
		\item[1.2] The starting node.
		\item[1.3] All the parameters leading from the starting node.
		\end{enumerate}
	\item The number of nodes we need to prove.
	\item The nodes that we need to prove.
	\end{enumerate}
\item
\mbox{The Specification of the output.}\linebreak
\linebreak
The output is a message to the console that can be either:

-The answer to the querry is true. This means there is at least 1 path that leads to "yes" for our nodes in toProve.

-The answer to the querry is false. This means there is no path that leads to "yes" for our nodes in toProve.
\pagebreak


\end{itemize}


\section{Conclusions}


\subsection{Achievements}
Understanding the logic behind some AI algorithms like RBFS and A* and how they pick their paths.
\subsection{Challenging and interesting	parts}
\sloppy
One interesting part of the project was understanding the heuristic sorting for the nodes so we can pick the best one to split next.
\linebreak
\subsection{Future directions for extending the project} 
One future direction for this project would be implementing it in a real world application or something like an AI in a game.
\pagebreak

\begin{thebibliography}{9}
	
	\bibitem{cormen09}
	Thomas H. Cormen and Charles E. Leiserson and Ronald L. Rivest and Clifford Stein,
	\emph{Introduction to Algorithms}.
	MIT Press,
	3rd Edition,
	2009.
	
	\bibitem{latex}
	\url{https://www.hackerrank.com/domains/ai/ai-introduction} \linebreak
	\url{http://artint.info/html/ArtInt.html}
	\linebreak
	\url{https://www.youtube.com/watch?v=EvvSYeI6BaQ}
	\linebreak
	\url{http://cs.gettysburg.edu/~tneller/papers/talks/RBFS_Example.htm}
		\linebreak
		\url{https://www.youtube.com/watch?v=tZYQywJJg9s}
		\linebreak
\end{thebibliography}



\end{document}